%------------------------------------------------------------------------------
% Beginning of journal.tex
%------------------------------------------------------------------------------
%
% AMS-LaTeX version 2 sample file for journals, based on amsart.cls.
%
%        ***     DO NOT USE THIS FILE AS A STARTER.      ***
%        ***  USE THE JOURNAL-SPECIFIC *.TEMPLATE FILE.  ***
%
% Replace amsart by the documentclass for the target journal, e.g., tran-l.
%
\documentclass{amsart}

%     If your article includes graphics, uncomment this command.
\usepackage{graphicx}
% move author affiliations around
\usepackage{amsaddr}
\usepackage{amsfonts}
\usepackage{amsmath}% 
\usepackage{calc}%     needed for the width/height calculations

\newtheorem{theorem}{Theorem}[section]
\newtheorem{lemma}[theorem]{Lemma}

\theoremstyle{definition}
\newtheorem{definition}[theorem]{Definition}
\newtheorem{example}[theorem]{Example}
\newtheorem{xca}[theorem]{Exercise}

\theoremstyle{remark}
\newtheorem{remark}[theorem]{Remark}

\numberwithin{equation}{section}

%    Absolute value notation
\newcommand{\abs}[1]{\lvert#1\rvert}

%    Blank box placeholder for figures (to avoid requiring any
%    particular graphics capabilities for printing this document).
\newcommand{\blankbox}[2]{%
  \parbox{\columnwidth}{\centering
%    Set fboxsep to 0 so that the actual size of the box will match the
%    given measurements more closely.
    \setlength{\fboxsep}{0pt}%
    \fbox{\raisebox{0pt}[#2]{\hspace{#1}}}%
  }%
}

% Variable definitions
\def\T{$\mathcal{T}$}
\def\TSolver{$\mathcal{T}$-\emph{solver}}
\def\sat{\texttt{SAT}}
\def\unsat{\texttt{UNSAT}}
% \newcommand{\eqdef}{\overset{\mathrm{def}}{=\joinrel=}}


% \newcommand*{\MyDef}{\mathrm{def}}
% \newcommand*{\eqdefU}{\ensuremath{\mathop{\overset{\MyDef}{=}}}}% Unscaled version
% \newcommand*{\eqdef}{\mathop{\overset{\MyDef}{\resizebox{\widthof{\eqdefU}}{\heightof{=}}{=}}}}

\newcommand\eqdef{\mathrel{\overset{\makebox[0pt]{\mbox{\normalfont\tiny\sffamily def}}}{=}}}


\begin{document}

\title[Satisfiability Modulo the Theory of Costs]{Satisfiability Modulo the Theory of Costs: Foundations and Applications}

%    Information for first author
\author{Marty Pye}
\address{RWTH Aachen University}
\email[Marty Pye]{marty.pye@rwth-aachen.de}

\begin{abstract}
Lorem ipsum dolor sit amet, consectetur adipiscing elit. Etiam diam sapien, egestas at velit in, tempor malesuada turpis. Aenean eget rutrum sapien. Proin a blandit justo. Nulla fermentum placerat lacus eu tempus. Quisque nec orci eros. Nullam tempor ut lectus elementum blandit. In sed velit quis massa viverra pellentesque. Donec bibendum nibh molestie, volutpat lectus sed, eleifend nibh. Donec nec tellus justo. Vestibulum nec massa ac sapien porta ultrices.

Mauris non consectetur quam. Suspendisse in orci consectetur, dictum erat nec, varius ante. Vivamus porta purus eget dictum dictum. Vivamus quis sem risus. Donec dictum tortor et odio tincidunt, non tincidunt nunc venenatis. Fusce nec lacus vel odio semper venenatis a sodales elit. Morbi tincidunt sit amet eros sed pharetra. Pellentesque varius pharetra ultrices. Pellentesque quam diam, blandit vel gravida in, vestibulum vel sapien. Quisque quam nibh, accumsan nec est placerat, eleifend vestibulum nulla. Cras a dapibus sapien, vel consequat urna. Cras bibendum fringilla mi dapibus tincidunt. Ut suscipit elit et risus porta, faucibus blandit ligula aliquet.
\end{abstract}

\maketitle

\section{Introduction}
\section{SMT Solving}
The \emph{Satisfiability Modulo Theories} (SMT($\mathcal{T}$)) problem is a decision problem for logic formulas under the background of a Theory $\mathcal{T}$, and can be seen as a constraint satisfaction problem.
A tool able to decide the satisfiability of sets of ground atomic formulas and their negations is referred to as a $\mathcal{T}$-\emph{solver}.
If the input set of \T{}-literals $\mu$ is satisfiable under \T{}, then the \TSolver{} returns \sat{}.
Additionally, the \TSolver{} can return a so-called \T{}-deduction clause. This can be used in early pruning, where the \T{}-deduction clause can be used in backjumping and learning.
If the input set of \T{}-literals is not satisfiable under \T{}, then the \TSolver{} returns \unsat{} and the conflict clause consisting of the subset of \T{}-literals in $\mu$ which was found \T{}-unsatisfiable.

A tool which is able to solve an SMT($\mathcal{T}$) problem is further referred to as an SMT($\mathcal{T}$)-solver.
In a lazy SMT(\T{})-solver, the truth assignments for $\varphi$ are checked for \T{}-satisfiability.
Usually, this is performed by a modified version of the DPLL algorithm.
If a truth assignment $\mu$ is found where $\mu \models \varphi$, then $\mu$ is passed on to the \TSolver{}.
If the \TSolver{} then returns \sat{}, the SMT(\T{})-solver found a solution.
If not, then the \TSolver{} passes the \T{}-conflict clause $\neg\eta$ back to the learning mechanism of the DPLL algorithm. For more information on how lazy SMT(\T{}) solving works, please refer to \cite{Sebastiani07}.

\section{Satisfiability Modulo the Theory of Costs}
Cimatti et al.\ extend the SMT framework by adding a possibility of modelling cost functions.
An SMT(\T{}) \emph{cost problem} is a pair $\langle \varphi, costs \rangle$, where $costs \eqdef \{cost^{i}\}^{M}_{i=1}$



The following is an example of a proof.

\begin{proof} Set $j(\nu)=\max(I\backslash a(\nu))-1$. Then we have
\[
\sum_{i\notin a(\nu)}t_i\sim t_{j(\nu)+1}
  =\prod^{j(\nu)}_{j=0}(t_{j+1}/t_j).
\]
Hence we have
\begin{equation}
\begin{split}
\prod_\nu\biggl(\sum_{i\notin
  a(\nu)}t_i\biggr)^{\abs{a(\nu-1)}-\abs{a(\nu)}}
&\sim\prod_\nu\prod^{j(\nu)}_{j=0}
  (t_{j+1}/t_j)^{\abs{a(\nu-1)}-\abs{a(\nu)}}\\
&=\prod_{j\ge 0}(t_{j+1}/t_j)^{
  \sum_{j(\nu)\ge j}(\abs{a(\nu-1)}-\abs{a(\nu)})}.
\end{split}
\end{equation}
By definition, we have $a(\nu(j))\supset c(j)$. Hence, $\abs{c(j)}=n-j$
implies (5.4). If $c(j)\notin a$, $a(\nu(j))c(j)$ and hence
we have (5.5).
\end{proof}

\begin{quotation}
This is an example of an `extract'. The magnetization $M_0$ of the Ising
model is related to the local state probability $P(a):M_0=P(1)-P(-1)$.
The equivalences are shown in Table~\ref{eqtable}.
\end{quotation}

\begin{table}[ht]
\caption{}\label{eqtable}
\renewcommand\arraystretch{1.5}
\noindent\[
\begin{array}{|c|c|c|}
\hline
&{-\infty}&{+\infty}\\
\hline
{f_+(x,k)}&e^{\sqrt{-1}kx}+s_{12}(k)e^{-\sqrt{-1}kx}&s_{11}(k)e^
{\sqrt{-1}kx}\\
\hline
{f_-(x,k)}&s_{22}(k)e^{-\sqrt{-1}kx}&e^{-\sqrt{-1}kx}+s_{21}(k)e^{\sqrt
{-1}kx}\\
\hline
\end{array}
\]
\end{table}

\begin{definition}
This is an example of a `definition' element.
For $f\in A(X)$, we define
\begin{equation}
\mathcal{Z} (f)=\{E\in Z[X]: \text{$f$ is $E^c$-regular}\}.
\end{equation}
\end{definition}

\begin{remark}
This is an example of a `remark' element.
For $f\in A(X)$, we define
\begin{equation}
\mathcal{Z} (f)=\{E\in Z[X]: \text{$f$ is $E^c$-regular}\}.
\end{equation}
\end{remark}

\begin{example}
This is an example of an `example' element.
For $f\in A(X)$, we define
\begin{equation}
\mathcal{Z} (f)=\{E\in Z[X]: \text{$f$ is $E^c$-regular}\}.
\end{equation}
\end{example}

\begin{xca}
This is an example of the \texttt{xca} environment. This environment is
used for exercises which occur within a section.
\end{xca}

The following is an example of a numbered list.

\begin{enumerate}
\item First item.
In the case where in $G$ there is a sequence of subgroups
\[
G = G_0, G_1, G_2, \dots, G_k = e
\]
such that each is an invariant subgroup of $G_i$.

\item Second item.
Its action on an arbitrary element $X = \lambda^\alpha X_\alpha$ has the
form
\begin{equation}\label{eq:action}
[e^\alpha X_\alpha, X] = e^\alpha \lambda^\beta
[X_\alpha X_\beta] = e^\alpha c^\gamma_{\alpha \beta}
 \lambda^\beta X_\gamma,
\end{equation}

\begin{enumerate}
\item First subitem.
\[
- 2\psi_2(e) =  c_{\alpha \gamma}^\delta c_{\beta \delta}^\gamma
e^\alpha e^\beta.
\]

\item Second subitem.
\begin{enumerate}
\item First subsubitem.
In the case where in $G$ there is a sequence of subgroups
\[
G = G_0, G_1, G_2, \ldots, G_k = e
\]
such that each subgroup $G_{i+1}$ is an invariant subgroup of $G_i$ and
each quotient group $G_{i+1}/G_{i}$ is abelian, the group $G$ is called
\textit{solvable}.

\item Second subsubitem.
\end{enumerate}
\item Third subitem.
\end{enumerate}
\item Third item.
\end{enumerate}

Here is an example of a cite. See.

\begin{theorem}
This is an example of a theorem.
\end{theorem}

\begin{theorem}[Marcus Theorem]
This is an example of a theorem with a parenthetical note in the
heading.
\end{theorem}

\section{Some more list types}
This is an example of a bulleted list.

\begin{itemize}
\item $\mathcal{J}_g$ of dimension $3g-3$;
\item $\mathcal{E}^2_g=\{$Pryms of double covers of $C=\openbox$ with
normalization of $C$ hyperelliptic of genus $g-1\}$ of dimension $2g$;
\item $\mathcal{E}^2_{1,g-1}=\{$Pryms of double covers of
$C=\openbox^H_{P^1}$ with $H$ hyperelliptic of genus $g-2\}$ of
dimension $2g-1$;
\item $\mathcal{P}^2_{t,g-t}$ for $2\le t\le g/2=\{$Pryms of double
covers of $C=\openbox^{C'}_{C''}$ with $g(C')=t-1$ and $g(C'')=g-t-1\}$
of dimension $3g-4$.
\end{itemize}

This is an example of a `description' list.

\begin{description}
\item[Zero case] $\rho(\Phi) = \{0\}$.

\item[Rational case] $\rho(\Phi) \ne \{0\}$ and $\rho(\Phi)$ is
contained in a line through $0$ with rational slope.

\item[Irrational case] $\rho(\Phi) \ne \{0\}$ and $\rho(\Phi)$ is
contained in a line through $0$ with irrational slope.
\end{description}

\bibliographystyle{amsplain}
\bibliography{references}

\end{document}

%------------------------------------------------------------------------------
% End of journal.tex
%------------------------------------------------------------------------------
